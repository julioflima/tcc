\chapter{Introdução}
\label{cap:introducao}

O Brasil enfrenta desafios econômicos estruturais que limitam seu desenvolvimento há décadas. Embora temas como corrupção, disputas políticas e polarização ideológica dominem o debate público e engajem intensamente a opinião popular, esses fenômenos são inerentes à natureza humana e ao processo político em si --- a corrupção nasce da autopreservação e de interesses pessoais, características intrínsecas à ação humana em sociedade. Paradoxalmente, o foco excessivo nesses temas desvia a atenção dos problemas verdadeiramente estruturantes: a infraestrutura deficiente, a baixa taxa de poupança nacional, o crédito historicamente caro e a falta de oportunidades que elevem o nível de qualidade de vida da população \cite{giambiagi2011}. Afinal, todo povo aspira à prosperidade, independentemente de sua posição no espectro político.

O sistema previdenciário brasileiro, composto principalmente pelo Instituto Nacional do Seguro Social (INSS) e pelo Fundo de Garantia do Tempo de Serviço (FGTS), foi concebido com o objetivo de garantir proteção social aos trabalhadores. Contudo, ao longo das décadas, esses mecanismos tornaram-se um peso fiscal significativo --- os gastos com previdência e assistência social representam aproximadamente 54\% das despesas primárias do governo federal --- limitando severamente a capacidade de investimento do Estado em áreas essenciais como educação, saúde e infraestrutura \cite{ifi2023}.

O modelo atual opera sob o regime de repartição simples, onde as contribuições dos trabalhadores ativos financiam os benefícios dos aposentados. Este sistema, embora solidário em sua concepção, apresenta vulnerabilidades frente ao envelhecimento populacional e às mudanças no mercado de trabalho. Além disso, os recursos do FGTS são direcionados ao governo, que os utiliza para financiamento habitacional e infraestrutura, oferecendo ao trabalhador rentabilidade inferior à inflação \cite{afonso2016}.

Diante desse cenário, este trabalho propõe uma alternativa inspirada no modelo australiano de aposentadoria (\textit{Superannuation}), adaptada à realidade brasileira e potencializada pela tecnologia blockchain. A proposta consiste em criar um mercado financeiro poupador que substitua gradualmente o INSS e FGTS, permitindo que os trabalhadores invistam diretamente em empresas nacionais, com controle compartilhado entre governo e cidadão, garantindo segurança jurídica e transparência.

\section*{Justificativa}

A relevância deste estudo reside na necessidade urgente de reformar o sistema previdenciário brasileiro, que se encontra em situação de insustentabilidade fiscal. As reformas implementadas até o momento têm sido paliativas, ajustando parâmetros como idade mínima e tempo de contribuição, sem alterar a estrutura fundamental do sistema.

A tecnologia blockchain oferece uma oportunidade única de implementar um sistema transparente, imutável e descentralizado, capaz de garantir a propriedade individual dos recursos previdenciários enquanto mantém mecanismos de controle que evitem a liquidação prematura ou uso inadequado dos fundos.

\section*{Objetivos}

O objetivo geral deste trabalho é propor um modelo de mercado financeiro poupador baseado em blockchain como alternativa ao sistema previdenciário atual (INSS e FGTS).

Os objetivos específicos são:
\begin{alineas}
    \item Analisar os problemas estruturais do sistema previdenciário brasileiro atual;
    \item Estudar o modelo australiano de \textit{Superannuation} como referência internacional;
    \item Propor uma arquitetura de sistema blockchain para gestão previdenciária;
    \item Identificar os benefícios e desafios da implementação do modelo proposto.
\end{alineas}

\section*{Estrutura do Trabalho}

Este trabalho está organizado da seguinte forma: o Capítulo \ref{cap:fundamentacao-teorica} apresenta a fundamentação teórica, abordando as escolas econômicas, o sistema previdenciário brasileiro, o modelo australiano e a tecnologia blockchain. O Capítulo \ref{cap:metodologia} descreve a metodologia utilizada. O Capítulo \ref{cap:resultados} apresenta a proposta detalhada do sistema. Por fim, o Capítulo \ref{cap:conclusao} traz as conclusões e sugestões para trabalhos futuros.
