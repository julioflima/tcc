\chapter{Fundamentação Teórica}
\label{cap:fundamentacao-teorica}

Este capítulo apresenta os fundamentos teóricos necessários para a compreensão da proposta deste trabalho. Inicialmente, são discutidas as duas principais correntes econômicas que influenciam o debate sobre política econômica brasileira. Em seguida, analisa-se o sistema previdenciário brasileiro atual. Posteriormente, apresenta-se o modelo australiano de aposentadoria como referência internacional. Por fim, introduz-se a tecnologia blockchain e suas aplicações potenciais.

\section{Escolas Econômicas e o Debate Brasileiro}

O debate econômico brasileiro é historicamente marcado pela contraposição entre duas visões distintas sobre o papel do Estado e os mecanismos de desenvolvimento econômico.

\subsection{Escola Austríaca de Economia}

A Escola Austríaca de Economia, cujos principais expoentes incluem Ludwig von Mises, Friedrich Hayek e Carl Menger, fundamenta-se em princípios que enfatizam o papel da poupança e do livre mercado no desenvolvimento econômico \cite{mises1949}.

Segundo esta corrente, a poupança é a base fundamental para o investimento produtivo. A acumulação de capital através da poupança voluntária permite o financiamento de projetos de longo prazo, aumentando a produtividade e, consequentemente, o padrão de vida da população. Para os austríacos, a interferência governamental no sistema de preços e na alocação de recursos gera distorções que resultam em má alocação de capital e ciclos econômicos \cite{hayek1944}.

A teoria austríaca defende que o livre mercado, através do sistema de preços, é o mecanismo mais eficiente para coordenar as decisões econômicas de milhões de indivíduos. A intervenção estatal, mesmo bem-intencionada, tende a gerar consequências não previstas que frequentemente agravam os problemas que pretendia resolver.

\subsection{Escola Desenvolvimentista}

Em contraposição, a escola desenvolvimentista, influenciada pelo pensamento keynesiano e cepalino, argumenta que países em desenvolvimento necessitam de intervenção estatal ativa para superar o subdesenvolvimento \cite{furtado1961}.

Esta corrente defende que o Estado deve atuar como indutor do desenvolvimento, coordenando investimentos estratégicos, protegendo a indústria nascente e redistribuindo renda para estimular o mercado interno. O desenvolvimentismo brasileiro influenciou fortemente as políticas econômicas desde a Era Vargas, com a criação de empresas estatais e mecanismos de fomento.

Os desenvolvimentistas argumentam que o livre mercado, em países periféricos, tende a perpetuar a condição de exportador de commodities e importador de manufaturados, impedindo o desenvolvimento industrial autônomo.

\subsection{Implicações para a Política Previdenciária}

A tensão entre estas duas visões manifesta-se claramente no debate previdenciário. A perspectiva austríaca tenderia a favorecer sistemas de capitalização individual, onde cada trabalhador acumula sua própria poupança para aposentadoria. Já a visão desenvolvimentista justifica sistemas de repartição solidária, onde o Estado coordena a redistribuição intergeracional.

O sistema brasileiro atual, baseado na repartição, reflete historicamente a influência desenvolvimentista. Contudo, as dificuldades fiscais crescentes têm levado a questionamentos sobre sua sustentabilidade de longo prazo.

\section{O Sistema Previdenciário Brasileiro}

O sistema previdenciário brasileiro é composto por três pilares principais: o Regime Geral de Previdência Social (RGPS), administrado pelo INSS; os Regimes Próprios de Previdência Social (RPPS), para servidores públicos; e a Previdência Complementar, de caráter facultativo.

\subsection{INSS: Estrutura e Funcionamento}

O Instituto Nacional do Seguro Social (INSS) é responsável pela administração do RGPS, que cobre a maioria dos trabalhadores brasileiros do setor privado. O sistema opera sob o regime de repartição simples, onde as contribuições correntes dos trabalhadores ativos financiam os benefícios dos aposentados e pensionistas \cite{gentil2006}.

As alíquotas de contribuição variam de 7,5\% a 14\% sobre o salário de contribuição, dependendo da faixa salarial, enquanto os empregadores contribuem com 20\% sobre a folha de pagamento. Estes recursos são destinados ao pagamento de aposentadorias por idade, por tempo de contribuição, aposentadorias especiais, pensões por morte e auxílios diversos.

O déficit do RGPS tem crescido consistentemente, atingindo valores superiores a R\$ 300 bilhões anuais. Este desequilíbrio decorre de fatores demográficos (envelhecimento populacional), estruturais (informalidade do mercado de trabalho) e políticos (regras de benefícios desconectadas da capacidade contributiva).

\subsection{FGTS: Histórico e Problemas}

O Fundo de Garantia do Tempo de Serviço (FGTS) foi criado em 1966 como alternativa à estabilidade decenal no emprego. Mensalmente, os empregadores depositam 8\% do salário do trabalhador em conta vinculada, que pode ser sacada em situações específicas como demissão sem justa causa, aposentadoria, compra de imóvel ou doenças graves.

Contudo, o FGTS apresenta problemas estruturais significativos. A rentabilidade do fundo, limitada à Taxa Referencial (TR) mais 3\% ao ano, historicamente ficou abaixo da inflação, resultando em perda real do poder de compra do trabalhador \cite{afonso2016}. Estudos indicam que, ao longo de décadas de trabalho, o trabalhador pode perder até 40\% do valor real de suas contribuições.

Os recursos do FGTS são utilizados pelo governo para financiar programas habitacionais (Minha Casa Minha Vida) e obras de infraestrutura e saneamento. Embora estes investimentos possam gerar benefícios sociais, representam uma transferência de riqueza dos trabalhadores para políticas públicas, sem a devida remuneração pelo custo de oportunidade.

\subsection{Impacto Fiscal e Sustentabilidade}

O sistema previdenciário brasileiro consome aproximadamente 13\% do PIB, percentual elevado para um país com estrutura demográfica ainda relativamente jovem. Projeções atuariais indicam que, sem reformas estruturais, este percentual pode ultrapassar 20\% nas próximas décadas \cite{tafner2019}.

O comprometimento de cerca de 70\% do orçamento federal com benefícios previdenciários e assistenciais limita severamente a capacidade de investimento público em áreas essenciais. Esta rigidez orçamentária cria um círculo vicioso: a falta de investimento em infraestrutura e educação reduz a produtividade, que por sua vez diminui a arrecadação e agrava o déficit previdenciário.

\section{O Modelo Australiano de Superannuation}

A Austrália implementou, a partir de 1992, um sistema de aposentadoria baseado em capitalização individual obrigatória, conhecido como \textit{Superannuation}. Este modelo é frequentemente citado como referência internacional de sucesso \cite{bateman2001}.

\subsection{Estrutura do Sistema}

O \textit{Superannuation} obriga empregadores a contribuir com um percentual do salário do empregado (atualmente 11\%, com previsão de aumento para 12\%) para um fundo de aposentadoria em nome do trabalhador. Os recursos são geridos por fundos de pensão privados, sujeitos a regulamentação governamental.

Os trabalhadores podem escolher entre diferentes fundos e estratégias de investimento, desde opções conservadoras até mais agressivas, de acordo com seu perfil de risco e horizonte temporal. Os recursos só podem ser acessados ao atingir a idade de preservação (atualmente entre 55 e 60 anos, dependendo da data de nascimento) ou em circunstâncias específicas como invalidez permanente.

\subsection{Resultados e Lições}

Após três décadas de operação, o sistema australiano acumulou mais de 3,5 trilhões de dólares australianos em ativos, equivalente a aproximadamente 170\% do PIB do país. Este volume de poupança de longo prazo contribuiu significativamente para o desenvolvimento do mercado de capitais australiano e para o financiamento de investimentos produtivos.

O modelo demonstra que sistemas de capitalização individual podem ser implementados com sucesso, desde que acompanhados de regulamentação adequada, transparência e mecanismos de proteção ao investidor. A diversificação de investimentos e a gestão profissional resultaram em rentabilidade média superior à inflação ao longo do tempo.

Contudo, o modelo também apresenta desafios, como a cobertura inadequada de trabalhadores informais, a complexidade do sistema para trabalhadores com baixa educação financeira e a exposição a riscos de mercado.

\section{Tecnologia Blockchain}

A tecnologia blockchain, popularizada inicialmente pelo Bitcoin em 2008, apresenta características que a tornam potencialmente aplicável a sistemas de registro e gestão de ativos previdenciários \cite{nakamoto2008}.

\subsection{Conceitos Fundamentais}

Blockchain é uma estrutura de dados distribuída que mantém um registro imutável e transparente de transações. As principais características incluem:

\begin{alineas}
    \item \textbf{Descentralização}: Os dados são replicados em múltiplos nós da rede, eliminando pontos únicos de falha;
    \item \textbf{Imutabilidade}: Uma vez registradas, as transações não podem ser alteradas ou excluídas;
    \item \textbf{Transparência}: Todas as transações são públicas e verificáveis por qualquer participante;
    \item \textbf{Segurança criptográfica}: As transações são protegidas por algoritmos criptográficos robustos.
\end{alineas}

\subsection{Smart Contracts e Multi-sig}

Smart contracts são programas autoexecutáveis armazenados na blockchain que executam automaticamente quando condições predefinidas são satisfeitas. Esta funcionalidade permite a automação de regras complexas de governança \cite{buterin2014}.

Carteiras multi-assinatura (multi-sig) requerem múltiplas chaves privadas para autorizar uma transação. Por exemplo, uma carteira 2-de-3 requer que pelo menos duas de três chaves autorizem qualquer movimentação. Este mecanismo permite implementar sistemas de controle compartilhado, onde nenhuma parte individual possui controle total sobre os recursos.

\subsection{Aplicações em Sistemas Financeiros}

Diversas aplicações de blockchain em sistemas financeiros têm sido desenvolvidas, incluindo moedas digitais de bancos centrais (CBDCs), sistemas de liquidação de ativos, registro de propriedade e identidade digital. Estas aplicações demonstram a viabilidade técnica de utilizar blockchain para gerenciar ativos de valor significativo com segurança e transparência.

A tokenização de ativos, processo de representar direitos de propriedade sobre ativos reais em tokens na blockchain, abre possibilidades para fracionamento de investimentos e aumento da liquidez de ativos tradicionalmente ilíquidos.

\section{Síntese}

A fundamentação teórica apresentada estabelece as bases para a proposta deste trabalho. O sistema previdenciário brasileiro enfrenta desafios fiscais estruturais que demandam soluções inovadoras. O modelo australiano demonstra a viabilidade de sistemas de capitalização individual. A tecnologia blockchain oferece mecanismos técnicos para implementar sistemas transparentes, seguros e com governança compartilhada. A convergência destes elementos fundamenta a proposta de um mercado financeiro poupador baseado em blockchain, a ser detalhada nos capítulos seguintes.
