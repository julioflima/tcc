\chapter{Proposta do Sistema}
\label{cap:resultados}

Este capítulo apresenta a proposta detalhada de um mercado financeiro poupador baseado em blockchain como alternativa ao sistema previdenciário atual. A proposta é estruturada em cinco componentes principais: arquitetura geral, mecanismo de contribuição e investimento, sistema de governança, regras operacionais e mecanismos de herança.

\section{Visão Geral da Proposta}

O sistema proposto visa criar um mercado financeiro poupador que substitua gradualmente o INSS e o FGTS, permitindo que os trabalhadores brasileiros acumulem patrimônio próprio através de investimentos em empresas nacionais. A proposta combina elementos do modelo australiano de Superannuation com as capacidades da tecnologia blockchain para criar um sistema transparente, seguro e com governança compartilhada.

Os princípios fundamentais do sistema são:
\begin{alineas}
    \item \textbf{Propriedade individual}: Os recursos pertencem ao trabalhador, não ao Estado;
    \item \textbf{Investimento produtivo}: Os recursos são direcionados a empresas reais, gerando retorno econômico;
    \item \textbf{Governança compartilhada}: Controle dividido entre trabalhador e governo, evitando uso indevido;
    \item \textbf{Transparência total}: Todas as transações são públicas e auditáveis;
    \item \textbf{Automatização}: Regras executadas por smart contracts, reduzindo burocracia.
\end{alineas}

\section{Arquitetura do Sistema}

A arquitetura proposta é composta por três camadas principais, conforme ilustrado na Figura \ref{fig:arquitetura}.

\begin{figure}[htb]
    \centering
    \caption{Arquitetura do Sistema Proposto}
    \label{fig:arquitetura}
    \begin{tikzpicture}[
        node distance=1.5cm,
        box/.style={rectangle, draw, minimum width=4cm, minimum height=1cm, align=center},
        arrow/.style={->, >=stealth, thick}
    ]
        % Camada de Usuário
        \node[box, fill=blue!20] (user) {Camada de Usuário\\(Interface Web/Mobile)};
        
        % Camada de Aplicação
        \node[box, fill=green!20, below=of user] (app) {Camada de Aplicação\\(Smart Contracts)};
        
        % Camada de Blockchain
        \node[box, fill=orange!20, below=of app] (chain) {Camada de Blockchain\\(Rede Distribuída)};
        
        % Conexões
        \draw[arrow] (user) -- (app);
        \draw[arrow] (app) -- (chain);
        
        % Atores externos
        \node[box, fill=gray!20, right=2cm of user] (trab) {Trabalhador};
        \node[box, fill=gray!20, right=2cm of app] (gov) {Governo};
        \node[box, fill=gray!20, right=2cm of chain] (emp) {Empresas};
        
        \draw[arrow] (trab) -- (user);
        \draw[arrow] (gov) -- (app);
        \draw[arrow] (emp) -- (chain);
    \end{tikzpicture}
    \fonte{Elaborado pelo autor (2026).}
\end{figure}

\subsection{Camada de Blockchain}

A camada base utiliza uma blockchain permissionada, operada por nós validadores distribuídos entre diferentes instituições (Banco Central, órgãos reguladores, grandes instituições financeiras). A escolha por blockchain permissionada, em vez de pública, justifica-se por:

\begin{alineas}
    \item Maior controle sobre quem pode validar transações;
    \item Melhor desempenho em termos de transações por segundo;
    \item Conformidade regulatória mais facilitada;
    \item Menor consumo energético comparado a redes de prova de trabalho.
\end{alineas}

\subsection{Camada de Smart Contracts}

Esta camada implementa toda a lógica de negócio do sistema através de smart contracts, incluindo:

\begin{alineas}
    \item \textbf{Contrato de Contribuição}: Recebe os depósitos mensais e registra na conta do trabalhador;
    \item \textbf{Contrato de Investimento}: Gerencia as ordens de compra e venda de ativos;
    \item \textbf{Contrato Multi-sig}: Implementa a governança compartilhada;
    \item \textbf{Contrato de Herança}: Gerencia a transferência de ativos em caso de falecimento;
    \item \textbf{Contrato de Compliance}: Verifica o cumprimento das regras operacionais.
\end{alineas}

\subsection{Camada de Usuário}

A interface de usuário consiste em aplicativo móvel e plataforma web que permitem ao trabalhador:

\begin{alineas}
    \item Visualizar seu saldo e histórico de contribuições;
    \item Consultar sua carteira de investimentos;
    \item Solicitar a operação mensal permitida;
    \item Acompanhar a rentabilidade de seus investimentos;
    \item Gerenciar beneficiários para herança.
\end{alineas}

\section{Mecanismo de Contribuição e Investimento}

\subsection{Fluxo de Contribuição}

O fluxo de contribuição proposto mantém a estrutura atual de desconto em folha, porém com destino diferente:

\begin{enumerate}
    \item O empregador desconta o percentual do salário do trabalhador (similar ao atual);
    \item O valor é convertido em tokens estáveis (stablecoins pareadas ao Real);
    \item Os tokens são depositados automaticamente na carteira multi-sig do trabalhador;
    \item O smart contract registra a transação e atualiza o saldo disponível para investimento.
\end{enumerate}

\subsection{Universo de Investimentos}

Os recursos podem ser investidos exclusivamente em ativos de empresas brasileiras, incluindo:

\begin{alineas}
    \item Ações de empresas listadas na B3;
    \item Debêntures de empresas brasileiras;
    \item Fundos de investimento em infraestrutura (FI-Infra);
    \item Fundos imobiliários (FIIs);
    \item Títulos públicos federais (como opção conservadora).
\end{alineas}

A restrição a ativos nacionais tem como objetivo:
\begin{alineas}
    \item Fomentar o mercado de capitais brasileiro;
    \item Financiar o crescimento de empresas nacionais;
    \item Evitar a evasão de divisas;
    \item Criar um ciclo virtuoso de poupança e investimento interno.
\end{alineas}

\subsection{Tokenização de Ativos}

Os ativos elegíveis seriam tokenizados, ou seja, representados por tokens na blockchain. Cada token representa uma fração do ativo subjacente, permitindo:

\begin{alineas}
    \item Investimentos fracionados, acessíveis a pequenos poupadores;
    \item Liquidação instantânea de operações;
    \item Registro imutável de propriedade;
    \item Auditabilidade completa das transações.
\end{alineas}

\section{Sistema de Governança Multi-sig}

O elemento central da proposta é o sistema de governança baseado em carteiras multi-assinatura, que implementa o controle compartilhado entre trabalhador e governo.

\subsection{Estrutura das Chaves}

Cada carteira previdenciária opera com um esquema 2-de-3, onde três chaves são necessárias, mas apenas duas são suficientes para autorizar transações:

\begin{alineas}
    \item \textbf{Chave do Trabalhador}: Controlada exclusivamente pelo titular da conta, armazenada em dispositivo pessoal seguro;
    \item \textbf{Chave do Governo}: Controlada por órgão regulador (ex: Banco Central ou nova autarquia), utilizada para co-assinar transações válidas;
    \item \textbf{Chave de Recuperação}: Armazenada em cofre seguro, para uso apenas em casos de perda de uma das outras chaves.
\end{alineas}

\subsection{Fluxo de Autorização}

Para realizar qualquer movimentação, o seguinte fluxo é executado:

\begin{enumerate}
    \item O trabalhador inicia a solicitação através do aplicativo, assinando com sua chave;
    \item O smart contract verifica se a operação está dentro das regras permitidas;
    \item Se válida, a solicitação é encaminhada ao sistema governamental;
    \item O sistema governamental co-assina automaticamente se todas as regras forem cumpridas;
    \item A transação é executada e registrada na blockchain.
\end{enumerate}

Este modelo garante que:
\begin{alineas}
    \item O governo sozinho não pode mover os recursos do trabalhador;
    \item O trabalhador sozinho não pode violar as regras do sistema;
    \item Há transparência total sobre todas as operações realizadas.
\end{alineas}

\subsection{Diagrama do Sistema Multi-sig}

A Figura \ref{fig:multisig} ilustra o funcionamento do sistema de governança multi-sig.

\begin{figure}[htb]
    \centering
    \caption{Sistema de Governança Multi-sig}
    \label{fig:multisig}
    \begin{tikzpicture}[
        node distance=2cm,
        key/.style={circle, draw, minimum size=1.5cm, align=center, fill=yellow!30},
        wallet/.style={rectangle, draw, minimum width=3cm, minimum height=2cm, align=center, fill=green!20},
        arrow/.style={->, >=stealth, thick}
    ]
        % Carteira central
        \node[wallet] (wallet) {Carteira\\Multi-sig\\(2 de 3)};
        
        % Chaves
        \node[key, above left=of wallet] (key1) {Chave\\Trabalhador};
        \node[key, above right=of wallet] (key2) {Chave\\Governo};
        \node[key, below=of wallet] (key3) {Chave\\Recuperação};
        
        % Conexões
        \draw[arrow] (key1) -- (wallet);
        \draw[arrow] (key2) -- (wallet);
        \draw[arrow, dashed] (key3) -- (wallet);
        
        % Legenda
        \node[right=3cm of wallet, align=left] {
            Linha contínua: uso regular\\
            Linha tracejada: uso emergencial
        };
    \end{tikzpicture}
    \fonte{Elaborado pelo autor (2026).}
\end{figure}

\section{Regras Operacionais}

Para garantir que o sistema cumpra seu objetivo previdenciário, são impostas regras operacionais rígidas, implementadas diretamente nos smart contracts.

\subsection{Restrição de Operações}

Cada trabalhador pode realizar no máximo uma operação de compra ou venda por mês. Esta restrição visa:

\begin{alineas}
    \item Evitar comportamento especulativo de curto prazo;
    \item Incentivar visão de longo prazo nos investimentos;
    \item Reduzir custos operacionais e de processamento;
    \item Simplificar a gestão da carteira pelo trabalhador médio.
\end{alineas}

\subsection{Bloqueio de Liquidação}

Os recursos são bloqueados até a aposentadoria, com exceções limitadas:

\begin{alineas}
    \item Aposentadoria por idade ou tempo de contribuição;
    \item Invalidez permanente comprovada;
    \item Doenças graves especificadas em lei;
    \item Falecimento (transferência para herdeiros).
\end{alineas}

Diferentemente do FGTS atual, não há saque para compra de imóvel ou demissão, reforçando o caráter estritamente previdenciário dos recursos.

\subsection{Limites de Concentração}

Para proteção do trabalhador, são impostos limites de diversificação:

\begin{alineas}
    \item Máximo de 10\% do patrimônio em uma única empresa;
    \item Máximo de 30\% em um único setor da economia;
    \item Mínimo de 20\% em ativos de baixo risco (títulos públicos ou equivalentes).
\end{alineas}

\subsection{Tabela de Regras Operacionais}

A Tabela \ref{tab:regras} resume as principais regras operacionais do sistema.

\begin{table}[htb]
    \centering
    \caption{Resumo das Regras Operacionais}
    \label{tab:regras}
    \begin{tabular}{p{5cm}p{8cm}}
        \toprule
        \textbf{Regra} & \textbf{Descrição} \\
        \midrule
        Operações mensais & Máximo 1 operação de compra/venda por mês \\
        Ativos elegíveis & Apenas empresas/ativos brasileiros \\
        Concentração máxima & 10\% por empresa, 30\% por setor \\
        Reserva obrigatória & Mínimo 20\% em baixo risco \\
        Saque & Apenas na aposentadoria ou exceções legais \\
        Governança & Multi-sig 2-de-3 (trabalhador + governo) \\
        \bottomrule
    \end{tabular}
    \fonte{Elaborado pelo autor (2026).}
\end{table}

\section{Mecanismo de Herança}

Um diferencial importante do sistema proposto é o tratamento da herança, que difere fundamentalmente do INSS atual.

\subsection{Herança Total do Patrimônio}

No sistema proposto, em caso de falecimento do titular, 100\% do patrimônio acumulado é transferido para os herdeiros designados. Isso contrasta com o sistema atual, onde o trabalhador que falece antes de se aposentar perde grande parte de suas contribuições.

O trabalhador pode, a qualquer momento, cadastrar seus beneficiários no sistema, especificando:

\begin{alineas}
    \item Nome e identificação dos beneficiários;
    \item Percentual destinado a cada beneficiário;
    \item Condições especiais (ex: liberação gradual para filhos menores).
\end{alineas}

\subsection{Processo de Transferência}

Em caso de falecimento:

\begin{enumerate}
    \item A família apresenta certidão de óbito ao sistema;
    \item O smart contract verifica a autenticidade do documento (integração com cartórios digitais);
    \item O patrimônio é transferido para novas carteiras multi-sig dos herdeiros;
    \item Os herdeiros assumem o controle, mantendo as mesmas regras operacionais;
    \item Herdeiros podem optar por liquidar gradualmente ou manter os investimentos.
\end{enumerate}

\subsection{Proteção contra Fraudes}

Para evitar fraudes, o sistema implementa:

\begin{alineas}
    \item Período de carência de 30 dias antes da transferência;
    \item Verificação cruzada com bases de óbito oficiais;
    \item Notificação a todos os beneficiários cadastrados;
    \item Possibilidade de contestação durante o período de carência.
\end{alineas}

\section{Transição do Sistema Atual}

A implementação do sistema proposto requer um período de transição cuidadosamente planejado.

\subsection{Estratégia de Transição Gradual}

Propõe-se uma transição de 30 anos:

\begin{alineas}
    \item \textbf{Anos 1-5}: Sistema piloto voluntário para novos entrantes no mercado de trabalho;
    \item \textbf{Anos 6-15}: Obrigatoriedade para novos trabalhadores, opcional para demais;
    \item \textbf{Anos 16-30}: Migração gradual dos trabalhadores em atividade que optarem pelo novo sistema;
    \item \textbf{Após ano 30}: INSS mantido apenas para aposentados atuais e trabalhadores que não migraram.
\end{alineas}

\subsection{Financiamento da Transição}

O principal desafio da transição é o financiamento dos benefícios atuais enquanto as contribuições são direcionadas ao novo sistema. Propõe-se:

\begin{alineas}
    \item Utilização de receitas do patrimônio público (royalties, dividendos de estatais);
    \item Redução gradual do déficit primário através de reformas administrativas;
    \item Emissão de títulos de longo prazo específicos para a transição previdenciária.
\end{alineas}

\section{Benefícios Esperados}

A implementação do sistema proposto traria benefícios em múltiplas dimensões:

\subsection{Para o Trabalhador}

\begin{alineas}
    \item Propriedade real sobre os recursos poupados;
    \item Rentabilidade potencialmente superior à inflação;
    \item Herança garantida para a família;
    \item Transparência total sobre aplicações e rendimentos.
\end{alineas}

\subsection{Para a Economia}

\begin{alineas}
    \item Aumento da taxa de poupança nacional;
    \item Desenvolvimento do mercado de capitais;
    \item Redução do custo de capital para empresas;
    \item Financiamento de investimentos produtivos.
\end{alineas}

\subsection{Para o Estado}

\begin{alineas}
    \item Eliminação gradual do déficit previdenciário;
    \item Liberação de recursos para investimentos públicos;
    \item Redução da carga tributária potencial no longo prazo;
    \item Modernização da infraestrutura financeira.
\end{alineas}

\section{Desafios e Riscos}

A proposta também apresenta desafios significativos:

\subsection{Desafios Técnicos}

\begin{alineas}
    \item Escalabilidade da blockchain para milhões de usuários;
    \item Segurança cibernética e proteção das chaves privadas;
    \item Integração com sistemas legados e bases governamentais;
    \item Educação financeira da população.
\end{alineas}

\subsection{Desafios Políticos}

\begin{alineas}
    \item Resistência de grupos beneficiados pelo sistema atual;
    \item Necessidade de reforma constitucional;
    \item Coordenação entre múltiplos órgãos governamentais;
    \item Comunicação efetiva com a população.
\end{alineas}

\subsection{Riscos de Mercado}

\begin{alineas}
    \item Exposição dos trabalhadores a volatilidade do mercado;
    \item Possibilidade de perdas em períodos de crise;
    \item Necessidade de mecanismos de proteção para trabalhadores próximos à aposentadoria.
\end{alineas}

\section{Arquitetura Descentralizada do Mercado}

Esta seção detalha como o mercado financeiro poupador pode ser concebido de maneira descentralizada, utilizando tecnologia blockchain para garantir transparência, segurança e governança distribuída.

\subsection{Camadas da Arquitetura Descentralizada}

O sistema é organizado em cinco camadas interdependentes, conforme ilustrado na Figura \ref{fig:camadas-descentralizadas}.

\begin{figure}[htb]
    \centering
    \caption{Arquitetura em Camadas do Mercado Descentralizado}
    \label{fig:camadas-descentralizadas}
    \begin{tikzpicture}[
        node distance=0.8cm,
        layer/.style={rectangle, draw, minimum width=12cm, minimum height=1.2cm, align=center, font=\small}
    ]
        \node[layer, fill=blue!15] (l1) {Camada de Interface (dApp Web3 / Mobile App)};
        \node[layer, fill=green!15, below=of l1] (l2) {Camada de Governança (MultiSigPowers.sol)};
        \node[layer, fill=yellow!15, below=of l2] (l3) {Camada de Mercado (RetirementVault + TradeController + DividendDistributor)};
        \node[layer, fill=orange!15, below=of l3] (l4) {Camada de Tokenização (RWAToken.sol - Ações B3 tokenizadas 1:1)};
        \node[layer, fill=red!15, below=of l4] (l5) {Camada de Oracle (Chainlink / Band Protocol - Preços B3 $\rightarrow$ Blockchain)};
    \end{tikzpicture}
    \fonte{Elaborado pelo autor (2026).}
\end{figure}

\subsubsection{Camada de Tokenização (RWA - Real World Assets)}

A tokenização de ativos reais é o fundamento do mercado descentralizado. Cada ação negociada na B3 é representada por um token ERC-20 na blockchain, mantendo paridade 1:1 com o ativo custodiado:

\begin{alineas}
    \item \textbf{Custódia Regulada}: Instituição custodiante autorizada pela CVM mantém as ações reais na CBLC (Central de Custódia e Liquidação);
    \item \textbf{Mint/Burn Controlado}: Tokens só podem ser criados (mint) quando ações são depositadas, e destruídos (burn) quando ações são resgatadas;
    \item \textbf{Auditoria Trimestral}: Verificação independente de que $\text{totalSupply}() = \text{ações custodiadas}$;
    \item \textbf{Pausabilidade}: Mecanismo de emergência para pausar operações em caso de inconsistências.
\end{alineas}

\subsubsection{Camada de Oracle}

Os oracles são responsáveis por sincronizar informações do mundo real (preços da B3) com a blockchain:

\begin{alineas}
    \item \textbf{Múltiplas Fontes}: Agregação de dados de B3, Bloomberg, Reuters e Yahoo Finance;
    \item \textbf{Mediana de Preços}: Utilização da mediana para resistir a manipulação de uma única fonte;
    \item \textbf{Heartbeat Check}: Atualização obrigatória a cada 15 minutos durante pregão;
    \item \textbf{Circuit Breaker}: Pausa automática se o desvio entre fontes exceder 10\%.
\end{alineas}

\subsubsection{Camada de Mercado}

O mercado pode ser implementado através de duas abordagens:

\textbf{Opção A - Automated Market Maker (AMM):} Pools de liquidez para cada par de ativos (ex: PETR4/BRL), onde o preço é determinado algoritmicamente pela proporção de ativos no pool. Vantagens: liquidez constante, operação 24/7.

\textbf{Opção B - Order Book On-Chain:} Livro de ordens descentralizado onde compradores e vendedores postam ordens limite. Vantagens: menor slippage para grandes volumes, descoberta de preço mais eficiente.

\subsection{Poderes do Smart Contract Multi-Sig}

O elemento central da governança é a definição precisa dos poderes de cada parte. A Tabela \ref{tab:poderes-multisig} apresenta a matriz de poderes implementada no smart contract.

\begin{table}[htb]
    \centering
    \caption{Matriz de Poderes do Smart Contract Multi-Sig}
    \label{tab:poderes-multisig}
    \begin{tabular}{p{4cm}ccp{4cm}}
        \toprule
        \textbf{Operação} & \textbf{Indivíduo} & \textbf{Governo} & \textbf{Condição} \\
        \midrule
        Trade mensal & \checkmark Exclusivo & $\times$ Bloqueado & Cooldown 30 dias \\
        Liquidação total & $\times$ Bloqueado & $\times$ Bloqueado & Idade $<$ 65 anos \\
        Liquidação total & \checkmark Livre & -- N/A & Idade $\geq$ 65 anos \\
        Adicionar herdeiro & \checkmark Exclusivo & $\times$ Bloqueado & Sempre \\
        Confisco judicial & -- N/A & \checkmark Exclusivo & Com ordem judicial \\
        Registrar óbito & -- N/A & \checkmark Exclusivo & Com certidão \\
        Distribuir herança & -- N/A & \checkmark Autoriza & Após óbito \\
        Liquidar estate & -- N/A & \checkmark Pleno & Após 100 anos \\
        Whitelist de ativos & $\times$ Bloqueado & \checkmark Exclusivo & Sempre \\
        \bottomrule
    \end{tabular}
    \fonte{Elaborado pelo autor (2026).}
\end{table}

\subsubsection{Poderes do Indivíduo}

O indivíduo (trabalhador) possui os seguintes poderes exclusivos:

\begin{alineas}
    \item \textbf{Executar Trades}: Pode comprar e vender ativos livremente dentro do universo de investimentos permitido, respeitando o limite de 1 operação por mês;
    \item \textbf{Gerenciar Beneficiários}: Tem controle exclusivo sobre quem são seus herdeiros e qual percentual cada um receberá;
    \item \textbf{Prova de Vida}: Deve submeter prova de vida anual para manter sua carteira ativa;
    \item \textbf{Plenos Poderes após 65 anos}: Ao atingir a idade de aposentadoria, pode liquidar qualquer posição sem necessidade de aprovação governamental.
\end{alineas}

É fundamental notar que o governo \textbf{não pode} fazer trades em nome do indivíduo. Esta restrição garante que a alocação de capital seja feita pela ``mão invisível'' das decisões individuais agregadas, não por uma autoridade central.

\subsubsection{Poderes do Governo}

O governo possui poderes específicos e limitados:

\begin{alineas}
    \item \textbf{Whitelist de Ativos}: Controla quais ativos podem ser negociados no sistema, garantindo que apenas empresas brasileiras reguladas participem;
    \item \textbf{Registro de Óbito}: Único autorizado a registrar falecimento no sistema, mediante apresentação de certidão;
    \item \textbf{Distribuição de Herança}: Autoriza a transferência de ativos para os beneficiários cadastrados;
    \item \textbf{Confisco Judicial}: Pode executar ordens judiciais de confisco, mas apenas com autorização do Poder Judiciário;
    \item \textbf{Plenos Poderes após 100 anos}: Em caso de ausência de prova de vida por 100 anos, assume controle total para liquidação do estate.
\end{alineas}

É crucial observar que o governo \textbf{não pode} confiscar ativos sem ordem judicial. O smart contract exige a role \texttt{JUDICIAL\_ROLE} além de \texttt{GOVERNMENT\_ROLE} para executar qualquer confisco.

\subsubsection{Checks and Balances}

O sistema implementa um mecanismo de freios e contrapesos onde:

\begin{alineas}
    \item O indivíduo sozinho não pode liquidar antes dos 65 anos (protege contra gastos impulsivos);
    \item O governo sozinho não pode mover os ativos (protege contra confisco arbitrário);
    \item O código é imutável após deploy (protege contra mudanças políticas);
    \item Todas as transações são públicas e auditáveis (garante transparência).
\end{alineas}

\subsection{Descentralização Progressiva}

A implementação do sistema pode seguir um caminho de descentralização progressiva:

\textbf{Fase 1 - Federada (Anos 1-5):} Governo opera os nós validadores, custodiantes regulados pela CVM, contratos auditados mas upgradáveis via timelock.

\textbf{Fase 2 - Híbrida (Anos 6-15):} Validadores distribuídos (40\% governo, 60\% entidades privadas), DAO para propostas de mudança, contratos com timelock de 30 dias para upgrades.

\textbf{Fase 3 - Descentralizada (Anos 16+):} Validadores eleitos por stake, governança 100\% on-chain, contratos imutáveis.

\subsection{Comparativo: Sistema Atual vs Sistema Proposto}

A Tabela \ref{tab:comparativo-sistemas} apresenta um comparativo entre as características do sistema atual (INSS/FGTS) e o sistema proposto baseado em blockchain.

\begin{table}[htb]
    \centering
    \caption{Comparativo entre Sistema Atual e Sistema Proposto}
    \label{tab:comparativo-sistemas}
    \begin{tabular}{p{3.5cm}p{4.5cm}p{4.5cm}}
        \toprule
        \textbf{Aspecto} & \textbf{Sistema Atual} & \textbf{Sistema Proposto} \\
        \midrule
        Transparência & Opaca, difícil auditoria & 100\% transparente, auditável \\
        Propriedade & Governo é dono dos recursos & Indivíduo é dono (com regras) \\
        Herança & Limitada por lei & Configurável pelo titular \\
        Rendimento & Abaixo da inflação (FGTS) & Mercado de ações brasileiro \\
        Portabilidade & Nenhuma & Global (chave privada) \\
        Resistência a mudanças & Sujeito a políticas & Código imutável é lei \\
        Custo operacional & Alto (burocracia estatal) & Baixo (automação) \\
        \bottomrule
    \end{tabular}
    \fonte{Elaborado pelo autor (2026).}
\end{table}

\subsection{Mitigação de Riscos da Descentralização}

A Tabela \ref{tab:riscos-mitigacao} apresenta os principais riscos do sistema descentralizado e suas respectivas mitigações.

\begin{table}[htb]
    \centering
    \caption{Riscos e Mitigações do Sistema Descentralizado}
    \label{tab:riscos-mitigacao}
    \begin{tabular}{p{4cm}p{8cm}}
        \toprule
        \textbf{Risco} & \textbf{Mitigação} \\
        \midrule
        Perda de chave privada & Social recovery wallets, custódia colaborativa, federações de custódia (Fedimint) \\
        Bug em smart contract & Múltiplas auditorias independentes, programa de bug bounty, deploy gradual \\
        Manipulação de oracle & Múltiplas fontes de dados, mediana, circuit breakers \\
        Colapso do mercado & Diversificação obrigatória, circuit breakers, reserva em ativos de baixo risco \\
        Ataque 51\% & Utilização de redes estabelecidas (Ethereum, Polygon) com alta segurança \\
        \bottomrule
    \end{tabular}
    \fonte{Elaborado pelo autor (2026).}
\end{table}
