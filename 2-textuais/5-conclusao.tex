\chapter{Conclusões e Trabalhos Futuros}
\label{cap:conclusao}

Este capítulo apresenta as conclusões do trabalho, sintetizando os principais resultados obtidos e indicando direções para trabalhos futuros.

\section{Conclusões}

Este trabalho propôs um modelo de mercado financeiro poupador baseado em blockchain como alternativa ao sistema previdenciário brasileiro atual, composto pelo INSS e FGTS. A proposta foi desenvolvida a partir da análise crítica do sistema vigente, do estudo do modelo australiano de Superannuation e das possibilidades oferecidas pela tecnologia blockchain.

O diagnóstico do sistema previdenciário brasileiro revelou problemas estruturais significativos. O INSS opera com déficits crescentes que comprometem o orçamento federal e limitam a capacidade de investimento do Estado. O FGTS oferece rentabilidade inferior à inflação, representando uma transferência silenciosa de riqueza dos trabalhadores para o financiamento de políticas públicas. Ambos os sistemas operam sob lógica de repartição ou gestão estatal, onde o trabalhador não possui propriedade real sobre suas contribuições.

O modelo australiano de Superannuation demonstrou que sistemas de capitalização individual podem ser implementados com sucesso, gerando acumulação significativa de poupança de longo prazo e contribuindo para o desenvolvimento do mercado de capitais. Após três décadas de operação, o sistema acumulou volume de recursos equivalente a 170\% do PIB australiano.

A tecnologia blockchain oferece mecanismos técnicos para implementar um sistema com características desejáveis: transparência através de registros públicos e auditáveis, segurança através de criptografia e descentralização, e governança compartilhada através de carteiras multi-assinatura.

A proposta desenvolvida combina estes elementos em um sistema onde:

\begin{alineas}
    \item Os trabalhadores são proprietários reais de seus recursos previdenciários;
    \item Os recursos são investidos em empresas brasileiras, fomentando o desenvolvimento nacional;
    \item A governança é compartilhada entre trabalhador e governo através de carteiras multi-sig 2-de-2;
    \item Regras operacionais restritivas garantem o uso previdenciário dos recursos;
    \item O patrimônio é integralmente herdável em caso de falecimento.
\end{alineas}

Os objetivos específicos propostos foram atingidos:

\begin{enumerate}
    \item Foi realizada análise dos problemas estruturais do sistema previdenciário brasileiro, identificando déficits crescentes, rentabilidade negativa do FGTS e ausência de propriedade individual;
    
    \item Foi estudado o modelo australiano de Superannuation, extraindo lições sobre estrutura de funcionamento, regulamentação e resultados de longo prazo;
    
    \item Foi proposta uma arquitetura de sistema blockchain para gestão previdenciária, incluindo camadas de blockchain, smart contracts e interface de usuário, com detalhamento do sistema de governança multi-sig;
    
    \item Foram identificados benefícios (propriedade individual, rentabilidade, herança, desenvolvimento do mercado de capitais) e desafios (escalabilidade técnica, resistência política, educação financeira) da implementação.
\end{enumerate}

A proposta representa uma mudança paradigmática na concepção do sistema previdenciário, transitando de um modelo onde o Estado administra recursos coletivos para um modelo onde o indivíduo é proprietário de sua poupança, com supervisão estatal para garantir o cumprimento das regras.

Reconhece-se que a implementação de tal proposta enfrentaria desafios significativos, incluindo resistência política de grupos beneficiados pelo sistema atual, necessidade de reformas constitucionais, desafios técnicos de escalabilidade e segurança, e a necessidade de ampla educação financeira da população.

Contudo, diante da trajetória insustentável do sistema atual, a busca por alternativas inovadoras torna-se imperativa. Este trabalho contribui para esse debate ao apresentar uma proposta estruturada que aproveita tecnologias emergentes para criar um sistema mais justo, transparente e sustentável.

\section{Trabalhos Futuros}

Este trabalho abre diversas possibilidades para pesquisas futuras:

\subsection{Desenvolvimento Técnico}

\begin{alineas}
    \item Implementação de um protótipo funcional do sistema em ambiente de teste (testnet);
    \item Análise de escalabilidade de diferentes plataformas blockchain para o volume esperado de transações;
    \item Desenvolvimento de interfaces de usuário acessíveis para diferentes perfis de trabalhadores;
    \item Estudo de mecanismos de segurança para proteção das chaves privadas dos usuários.
\end{alineas}

\subsection{Análise Quantitativa}

\begin{alineas}
    \item Simulação do impacto fiscal da transição em diferentes cenários;
    \item Modelagem atuarial comparando resultados projetados do novo sistema versus manutenção do sistema atual;
    \item Análise de sensibilidade a diferentes parâmetros (alíquotas, rentabilidade, tempo de transição).
\end{alineas}

\subsection{Aspectos Jurídicos e Regulatórios}

\begin{alineas}
    \item Análise detalhada das alterações constitucionais e legais necessárias;
    \item Estudo comparativo de marcos regulatórios de outros países que implementaram reformas similares;
    \item Proposta de arcabouço regulatório para custódia e negociação de ativos tokenizados.
\end{alineas}

\subsection{Aspectos Sociais}

\begin{alineas}
    \item Pesquisa de campo sobre percepção e aceitação da proposta pela população;
    \item Desenvolvimento de programas de educação financeira adequados ao público-alvo;
    \item Estudo de mecanismos de proteção para trabalhadores vulneráveis durante a transição.
\end{alineas}

\subsection{Extensões do Modelo}

\begin{alineas}
    \item Integração com sistema de identidade digital nacional;
    \item Possibilidade de portabilidade internacional para trabalhadores migrantes;
    \item Aplicação do modelo para outros tipos de poupança compulsória;
    \item Estudo de mecanismos de seguro coletivo para proteção contra riscos de mercado.
\end{alineas}

A continuidade destas pesquisas poderá contribuir para amadurecer a proposta e eventualmente viabilizar sua implementação, contribuindo para a construção de um sistema previdenciário mais justo, sustentável e alinhado com as tecnologias do século XXI.
