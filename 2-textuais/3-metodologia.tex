\chapter{Metodologia}
\label{cap:metodologia}

Este capítulo apresenta os procedimentos metodológicos adotados para o desenvolvimento deste trabalho, incluindo a classificação da pesquisa, os métodos utilizados e as etapas de desenvolvimento da proposta.

\section{Classificação da Pesquisa}

Quanto à natureza, esta pesquisa classifica-se como aplicada, pois visa gerar conhecimentos para aplicação prática, direcionados à solução de um problema específico: a insustentabilidade do sistema previdenciário brasileiro \cite{gil2008}.

Quanto aos objetivos, a pesquisa é exploratória e propositiva. É exploratória porque busca proporcionar maior familiaridade com o problema, tornando-o mais explícito através da análise de diferentes modelos previdenciários e tecnologias disponíveis. É propositiva porque culmina na apresentação de uma proposta de solução estruturada.

Quanto à abordagem, adota-se a perspectiva qualitativa, caracterizada pela análise interpretativa e contextualizada de fontes primárias e secundárias — incluindo legislação vigente, relatórios de organismos internacionais, dados econômicos públicos e estudos comparados de modelos previdenciários estrangeiros. Diferentemente de abordagens quantitativas, que privilegiam a mensuração estatística de variáveis, a pesquisa qualitativa permite explorar as relações conceituais entre sistemas complexos e propor soluções fundamentadas em princípios teóricos, ainda que não empiricamente testadas em larga escala \cite{creswell2014}.

\section{Procedimentos Técnicos}

Os procedimentos técnicos utilizados incluem:

\subsection{Pesquisa Bibliográfica}

A pesquisa bibliográfica foi conduzida em bases de dados acadêmicas, incluindo Google Scholar, Scielo, Web of Science e repositórios institucionais. Foram consultadas obras clássicas de economia (Escola Austríaca e Desenvolvimentista), estudos sobre previdência social, documentação técnica sobre blockchain e relatórios de organismos internacionais.

Os termos de busca incluíram: ``reforma previdenciária'', ``sistema de capitalização'', ``superannuation'', ``blockchain financial applications'', ``multi-signature wallets'', ``INSS déficit'', ``FGTS rentabilidade'', entre outros.

\subsection{Pesquisa Documental}

Foram analisados documentos oficiais, incluindo:
\begin{alineas}
    \item Legislação previdenciária brasileira (Lei 8.213/91, Emendas Constitucionais 20/98, 41/03 e 103/19);
    \item Relatórios do Tesouro Nacional sobre resultado previdenciário;
    \item Documentação do sistema australiano de Superannuation;
    \item Whitepapers e documentação técnica de protocolos blockchain.
\end{alineas}

\subsection{Análise Comparativa}

Foi realizada análise comparativa entre o modelo brasileiro atual e o modelo australiano de Superannuation, identificando semelhanças, diferenças, pontos fortes e fragilidades de cada sistema. Esta análise fundamentou a identificação de elementos adaptáveis à realidade brasileira.

\section{Etapas de Desenvolvimento}

O desenvolvimento deste trabalho seguiu as seguintes etapas:

\subsection{Etapa 1: Diagnóstico do Sistema Atual}

Nesta etapa, foi realizado um levantamento detalhado do funcionamento do INSS e FGTS, incluindo:
\begin{alineas}
    \item Estrutura de contribuições e benefícios;
    \item Evolução histórica do déficit;
    \item Projeções atuariais;
    \item Identificação dos principais problemas estruturais.
\end{alineas}

\subsection{Etapa 2: Estudo de Modelos Internacionais}

Foi estudado o modelo australiano de Superannuation como principal referência, analisando:
\begin{alineas}
    \item Estrutura de funcionamento;
    \item Resultados após três décadas de operação;
    \item Lições aprendidas e desafios;
    \item Elementos adaptáveis ao contexto brasileiro.
\end{alineas}

\subsection{Etapa 3: Estudo da Tecnologia Blockchain}

Foram estudados os fundamentos técnicos de blockchain relevantes para a proposta:
\begin{alineas}
    \item Arquitetura de redes distribuídas;
    \item Mecanismos de consenso;
    \item Smart contracts e sua programabilidade;
    \item Carteiras multi-assinatura e governança;
    \item Casos de uso em sistemas financeiros.
\end{alineas}

\subsection{Etapa 4: Elaboração da Proposta}

Com base nas etapas anteriores, foi elaborada a proposta de sistema, contemplando:
\begin{alineas}
    \item Arquitetura geral do sistema;
    \item Mecanismos de contribuição e investimento;
    \item Sistema de governança com chaves multi-sig;
    \item Regras de operação e restrições;
    \item Mecanismos de herança e sucessão;
    \item Análise de viabilidade e desafios.
\end{alineas}

\subsection{Etapa 5: Validação Conceitual}

A proposta foi validada conceitualmente através de:
\begin{alineas}
    \item Verificação de consistência interna;
    \item Análise de aderência aos objetivos propostos;
    \item Identificação de limitações e trabalhos futuros.
\end{alineas}

\section{Limitações Metodológicas}

Este trabalho apresenta limitações inerentes à sua natureza propositiva:

\begin{alineas}
    \item Não foi implementado um protótipo funcional do sistema proposto;
    \item Não foram realizadas simulações quantitativas de impacto fiscal;
    \item A proposta não aborda aspectos jurídicos detalhados de implementação;
    \item Não foram conduzidas pesquisas de campo sobre aceitação social da proposta.
\end{alineas}

Estas limitações apontam direções para trabalhos futuros que possam aprofundar e validar empiricamente a proposta apresentada.
