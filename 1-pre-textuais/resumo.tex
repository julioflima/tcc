\begin{resumo}

O sistema previdenciário brasileiro, composto pelo Instituto Nacional do Seguro Social (INSS) e pelo Fundo de Garantia do Tempo de Serviço (FGTS), enfrenta desafios estruturais que comprometem sua sustentabilidade fiscal e a proteção efetiva dos trabalhadores. O déficit previdenciário consome parcela significativa do orçamento federal, enquanto o FGTS oferece rentabilidade historicamente inferior à inflação. Este trabalho propõe um modelo alternativo de mercado financeiro poupador baseado em tecnologia blockchain, inspirado no sistema australiano de \textit{Superannuation}. A proposta consiste em um sistema onde os trabalhadores investem suas contribuições diretamente em empresas brasileiras, com governança compartilhada através de carteiras multi-assinatura (multi-sig) que requerem autorização conjunta do trabalhador e do governo para movimentações. O modelo preserva a propriedade individual dos recursos, garante a herança integral do patrimônio, promove transparência total através de registros em blockchain e restringe operações especulativas ao permitir apenas uma transação mensal. A fundamentação teórica aborda as escolas econômicas Austríaca e Desenvolvimentista, analisa o sistema previdenciário brasileiro, estuda o modelo australiano e introduz conceitos de blockchain. A metodologia adotada é de pesquisa exploratória e qualitativa, baseada em revisão bibliográfica e análise documental. Os resultados apresentam a arquitetura detalhada do sistema, incluindo camadas de blockchain, smart contracts e interface de usuário, além de regras operacionais e mecanismos de herança. Conclui-se que a proposta oferece uma alternativa viável aos problemas identificados, embora sua implementação enfrente desafios técnicos, políticos e sociais que demandam estudos futuros.

\vspace{\onelineskip}
\noindent\textbf{Palavras-chave}: Previdência Social. Blockchain. Multi-assinatura. INSS. FGTS. Superannuation. Mercado de Capitais.
\end{resumo}
