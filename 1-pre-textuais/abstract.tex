\begin{otherlanguage*}{english}

The Brazilian social security system, comprising the National Social Security Institute (INSS) and the Severance Indemnity Fund (FGTS), faces structural challenges that compromise its fiscal sustainability and effective worker protection. The pension deficit consumes a significant portion of the federal budget, while FGTS has historically offered returns below inflation. This work proposes an alternative savings-based financial market model built on blockchain technology, inspired by the Australian Superannuation system. The proposal consists of a system where workers invest their contributions directly in Brazilian companies, with shared governance through multi-signature (multi-sig) wallets that require joint authorization from both worker and government for any transactions. The model preserves individual ownership of resources, guarantees full inheritance of assets, promotes total transparency through blockchain records, and restricts speculative operations by allowing only one transaction per month. The theoretical foundation addresses the Austrian and Developmentalist economic schools, analyzes the Brazilian pension system, studies the Australian model, and introduces blockchain concepts. The methodology adopted is exploratory and qualitative research, based on literature review and document analysis. The results present the detailed system architecture, including blockchain layers, smart contracts, and user interface, as well as operational rules and inheritance mechanisms. It is concluded that the proposal offers a viable alternative to the identified problems, although its implementation faces technical, political, and social challenges that require further studies.

\vspace{\onelineskip}
\noindent\textbf{Keywords}: Social Security. Blockchain. Multi-signature. INSS. FGTS. Superannuation. Capital Markets.
\end{otherlanguage*}
